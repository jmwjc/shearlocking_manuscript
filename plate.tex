\section{Reissner--Mindlin plate formulation}

\subsection{Governing equations for Mindlin plate}

Consider a thick plate $\Omega\times [-\frac{t}{2},\frac{t}{2}]$ with its mid-surface $\Omega$ and thickness $t$ in a Cartesian coordinate system, as shown in Figure, where $x_\alpha,\; \alpha=1,2$ are the in-plane coordinates on $\Omega$ and $x_3$ is the coordinate for the thickness direction. 
In accordance with Reissner-Mindlin hypothesis and inconsideration of the in-plane displacement of $\Omega$, the displacement $u_i,\; i = 1,2,3$ are given as follows:
\begin{equation} 
    \left\{
    \begin{aligned}
        u_\alpha(\boldsymbol x) &= - x_3 \varphi_\alpha(x_1, x_2) \\
        u_3(\boldsymbol x) &= w(x_1, x_2)
    \end{aligned}
    \right.
    , \quad \alpha = 1, 2
\end{equation}
where $w$ stands for the deflection and $\varphi_\alpha$ presents $\alpha$th direction component for rotation $\boldsymbol \varphi$.
The corresponding infinitesimal strain components $\varepsilon_{ij}$ are listed as follows:
\begin{equation}
    \left\{
    \begin{aligned}
        \varepsilon_{\alpha\beta} &= \frac{1}{2} (u_{\alpha,\beta} + u_{\beta,\alpha})
        = x_3 \kappa_{\alpha\beta} \\
        \varepsilon_{\alpha 3} &= \frac{1}{2} \gamma_\alpha \\
        \varepsilon_{33} &= 0
    \end{aligned}
    \right.
\end{equation}
in which $\kappa_{\alpha\beta}$ and $\gamma_\alpha$ are the components for the in-plane curvature $\boldsymbol \kappa$ and shear strain $\boldsymbol \gamma$ respectively, the $\boldsymbol \kappa$ and $\boldsymbol \gamma$ are defined as:
\begin{align}
    \boldsymbol \kappa &= - \frac{1}{2} (\nabla \boldsymbol \varphi+\boldsymbol \varphi \nabla )
    \\
    \boldsymbol \gamma &= \nabla w - \boldsymbol \varphi
\end{align}

According to the isotropic Mindlin plate theory \cite{hughes2000}, the stress components $\sigma_{ij}$ are given as follows:
\begin{equation}
    \left\{
    \begin{aligned}
        \sigma_{\alpha\beta} &= D_{\alpha\beta\gamma\eta} \varepsilon_{\gamma\eta} \\
        \sigma_{\alpha 3} &= G \gamma_\alpha \\
        \sigma_{33} &= 0
    \end{aligned}
    \right.
\end{equation}
where $D_{\alpha\beta\gamma\eta}$ is denoted as the component for the fourth-order in-plane constitutive tensor $\boldsymbol D$ and $G$ is shear modulus, and they can be evaluated by Young's modulus $E$, Poisson's ratio $\nu$ as:
\begin{align}
    \boldsymbol D &=
    \frac{E}{1-\nu^2} \left (
        \nu \boldsymbol 1\otimes \boldsymbol 1 + \frac{1}{2}(1-\nu) \boldsymbol I
    \right )
    \\
    G &= \frac{E}{2(1+\nu)}
\end{align}
in which $\boldsymbol 1 = \delta_{\alpha\beta} \boldsymbol e_i\otimes \boldsymbol e_j$ and
$\boldsymbol I = \delta_{\alpha\gamma}\delta_{\beta\eta} + \delta_{\alpha\eta}\delta_{\beta\gamma} \boldsymbol e_\alpha \otimes \boldsymbol e_\beta \otimes \boldsymbol e_\gamma \otimes \boldsymbol e_\eta$ 
represent the second- and fourth-order identity tensors,
and $\delta_{\alpha\beta}$ is Kronecker delta function. 
Further integrating the stress components through the thickness direction, we can obtain the moment resultant $\boldsymbol m$ and shear force resultant $\boldsymbol q$, given as follows:
\begin{align}
    \boldsymbol m &= \int_{-t/2}^{t/2} \sigma_{\alpha\beta} dx_3 \boldsymbol e_\alpha \otimes \boldsymbol e_\beta = 
    \frac{t^3}{12} \boldsymbol D : \boldsymbol \kappa \\
    \boldsymbol q &= k\int_{-t/2}^{t/2} \sigma_{\alpha 3} dx_3 \boldsymbol e_\alpha =
    kGt \boldsymbol \gamma
\end{align}
where $k$ presents the shear correction factor that is chosen as $k=\frac{4}{5}$ in this work. 
It should be noted that, the geometric parameter for the moment resultant has two more order power of thickness than shear force resultant.
As the thickness reducing, the parameter for the moment resultant will much less than that for shear force resultant,
for a specific external force, leading to a constraint of shear strain trending to be zero, i.e. $t\rightarrow 0, \boldsymbol \gamma \rightarrow \boldsymbol 0$. 

Consequently, the governing equations for Reissner-Mindlin plate can be expressed as follows:
\begin{equation}
\left \{
\begin{aligned}
\nabla \cdot \boldsymbol m - \boldsymbol q &= \boldsymbol 0 & &\textrm{in}\; \Omega \\
\nabla \cdot \boldsymbol q + \bar p &= 0 & &\textrm{in}\; \Omega \\
\boldsymbol q \cdot \boldsymbol  n &= \bar q_n & &\textrm{on}\; \Gamma_q \\
\boldsymbol m \cdot \boldsymbol n &= \bar{\boldsymbol m} & &\textrm{on}\; \Gamma_m \\
\boldsymbol \varphi &= \bar{\boldsymbol \varphi} & &\textrm{on}\; \Gamma_\varphi \\
w &= \bar w & &\textrm{on}\; \Gamma_w \\
\end{aligned}
\right .
\end{equation}
where $\bar p$ is the transverse load,
$\bar q_n$ and $\bar{\boldsymbol m}$ are the prescribed shear force and moment on the natural boundary $\Gamma_q$ and $\Gamma_m$ respectively,
$\bar w$ and $\bar{\boldsymbol \varphi}$ are the prescribed deflection and rotation on the essential boundary $\Gamma_w$ and $\Gamma_\varphi$ respectively,
and $\boldsymbol n$ is the unit outward normal vector on the boundary.
All the boundary segments satisfy the following relationship:
\begin{equation}
\Gamma_q \cup \Gamma_m = \Gamma_w \cup \Gamma_\varphi = \Gamma
,\quad
\Gamma_w \cap \Gamma_\varphi = \Gamma_q \cap \Gamma_m = \varnothing
\end{equation}
and $\Gamma$ is the boundary of $\Omega$.


\subsection{Mixed--formulation}

The thick plate Galerkin formulation generally use the three-variable weak form to immigrate the locking issue. 
To obtain the corresponding Galerkin weak form, the complementary energy functional is given by: 
\begin{equation}
    \begin{aligned}
        \Pi(\boldsymbol u, \boldsymbol q) &=
        \int_\Omega \frac{t^3}{24} \boldsymbol \kappa : \boldsymbol D : \boldsymbol \kappa d\Omega - \int_\Omega \frac{1}{2kGt} \boldsymbol q \cdot \boldsymbol q d\Omega - \int_\Omega \boldsymbol \varphi \cdot \boldsymbol q d\Omega \\
        &+ \int_{\Gamma_w} \boldsymbol q \cdot \boldsymbol n \bar w d\Gamma + \int_{\Gamma_m} \boldsymbol \varphi \cdot \bar{\boldsymbol m} d\Gamma + \int_{\Gamma_q} w (\boldsymbol q \cdot \boldsymbol n - \bar q_n) d\Gamma \\&- \int_\Omega w (\nabla \cdot \boldsymbol q + \bar p) d\Omega
    \end{aligned}
\end{equation}
where $\boldsymbol u$ stands for a displacement set containing deflection $w$ and rotation $\boldsymbol \varphi$, i.e. $\boldsymbol u = \{w,\boldsymbol \varphi\}$.
Introducing the standard variational operation herein leads to the following Galerkin weak form:
find $w \in W$, $\boldsymbol \varphi \in \Phi$, $\boldsymbol q \in Q$,
\begin{equation}\label{eq:weak_form}
\begin{aligned}
a(\delta \boldsymbol \varphi, \boldsymbol \varphi) + b_{\varphi}(\delta\boldsymbol \varphi,\boldsymbol q) &= f_{\varphi}(\delta \boldsymbol \varphi), & \forall \delta \boldsymbol \varphi \in \Phi \\
b_{w}(\delta w,\boldsymbol q) &= f_{w}(\delta w), & \forall \delta w \in W \\
b(\boldsymbol u,\delta \boldsymbol q) &= f_{q}(\delta \boldsymbol q), & \forall \delta \boldsymbol q \in Q
\end{aligned}
\end{equation}
where the deflection space $W$, rotation space $\Phi$ and shear force space $Q$ are defined as:
\begin{align}
    W &:= \{\} \\
    \Phi &:= \{\} \\
    Q &:= \{\}
\end{align}
and the bilinear forms and linear forms in Eq. \eqref{eq:weak_form} are detailed as follows: 
\begin{align}
% &a: \Phi \times \Phi \rightarrow \mathbb R &
a(\delta\boldsymbol \varphi,\boldsymbol \varphi) &= \int_\Omega \delta \boldsymbol \kappa : \frac{t^3}{12} \boldsymbol D : \boldsymbol \kappa d\Omega \\
% &b: (W\times \Phi)\times Q \rightarrow \mathbb R &
b(\boldsymbol u, \boldsymbol q) &= b_w(w, \boldsymbol q) + b_\varphi(\boldsymbol \varphi, \boldsymbol q) \\
% &b_\varphi : \Phi\times Q \rightarrow \mathbb R &
b_{\varphi}(\boldsymbol \varphi, \boldsymbol q) &= -\int_{\Omega}\boldsymbol \varphi \cdot \boldsymbol q d\Omega \\
% &b_w : W\times Q \rightarrow \mathbb R &
\begin{split}
b_{w}(w,\boldsymbol q) &= \int_{\Gamma_{q}} w \boldsymbol q \cdot \boldsymbol nd\Gamma - \int_{\Omega}w \nabla \cdot \boldsymbol qd\Omega \\
&= \int_\Omega \nabla w \cdot \boldsymbol q d\Omega - \int_{\Gamma_w} w \boldsymbol q \cdot \boldsymbol n d\Gamma
\end{split} \\
% &f_\varphi: \Phi \rightarrow \mathbb R &
f_{\varphi}(\boldsymbol \varphi) &= - \int_{\Gamma_{m}} \boldsymbol \varphi \cdot \bar{\boldsymbol m}d\Gamma \\
% &f_w: W\rightarrow \mathbb R &
f_{w}(w) &= \int_{\Gamma_{q}} w \bar{q}_n d\Gamma + \int_{\Omega}w \bar{p} d\Omega \\
% &f_q: Q\rightarrow \mathbb R &
f_{q}(\boldsymbol q) &= - \int_{\Gamma_{w}} \boldsymbol q \cdot \boldsymbol n \bar{w} d\Gamma
\end{align}

Furthermore, we introduce the approximation schemes to discrete these three variables, namely $w_h$, $\boldsymbol \varphi_h$, $\boldsymbol q_h$, these approximants are represented by shape functions $N_I^w$, $N_K^\varphi$, $N_M^q$ and their corresponding nodal coefficients $w_I$, $\boldsymbol \varphi_K$, $\boldsymbol q_M$ respectively, yields:
\begin{equation}\label{ep:approximant}
    \left \{
    \begin{aligned}
        w_h(\boldsymbol x) &= \sum_{I=1}^{n_w} N_I^w(\boldsymbol x) w_I \\
        \boldsymbol \varphi_h(\boldsymbol x) &= \sum_{K=1}^{n_\varphi} N_K^\varphi \boldsymbol \varphi_K \\
        \boldsymbol q_h(\boldsymbol x) &= \sum_{R=1}^{n_q} N_R^q(\boldsymbol x) \boldsymbol q_R
    \end{aligned}
    \right .
\end{equation}
Substituting Eq. \eqref{ep:approximant} back into Eq. \eqref{eq:weak_form} can obtain the Galerkin weak formulation:
find $w_h \in W_h$, $\boldsymbol \varphi_h \in \Phi_h$, $\boldsymbol q_h \in Q_h$,
\begin{equation}\label{eq:weak_form_discrete}
\begin{aligned}
a(\delta \boldsymbol \varphi_h, \boldsymbol \varphi_h) + b_{\varphi}(\delta\boldsymbol \varphi_h,\boldsymbol q_h) &= f_{\varphi}(\delta \boldsymbol \varphi_h), & \forall \delta \boldsymbol \varphi_h \in \Phi_h \\
b_{w}(\delta w_h,\boldsymbol q_h) &= f_{w}(\delta w_h), & \forall \delta w_h \in W_h \\
b(\boldsymbol u_h,\delta \boldsymbol q_h) &= f_{q}(\delta \boldsymbol q_h), & \forall \delta \boldsymbol q_h \in Q_h
\end{aligned}
\end{equation}
where $\boldsymbol u_h = \{w_h, \boldsymbol \varphi_h\}$. The related approximation spaces are defined as:
\begin{align}
    W_h &:= \mathrm{span} \{N_I\}_{I=1}^{n_w} \\
    \Phi_h &:= \mathrm{span} \{N_K\}_{K=1}^{n_\varphi} \\
    Q_h &:= \mathrm{span} \{N_R\}_{R=1}^{n_q}
\end{align}
FIXME: approximation types

\begin{rmk}
    As shown in Eq. \eqref{eq:weak_form_discrete}, there are two constraints in thick plate mixed formulation. One, as listed in third line of Eq. \eqref{eq:weak_form_discrete}, is a mixed constraint regarding the shear in entire domain and deflection on essential deflection boundary. 
    The other is the transverse force enforcement in whole domain.
    However, for most of pinoneer work, only consider the shear constraint, and that is not sufficient for a satisfaction result.
\end{rmk}

\begin{rmk}

\end{rmk}
